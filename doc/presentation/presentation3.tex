\documentclass{beamer}

\usepackage{listings}
\usepackage{syntax}
\usepackage{amsmath}

\lstset{
  frame=none,
  xleftmargin=2pt,
  stepnumber=1,
  numbers=left,
  numbersep=5pt,
  numberstyle=\ttfamily\tiny\color[gray]{0.3},
  belowcaptionskip=\bigskipamount,
  captionpos=b,
  escapeinside={*'}{'*},
%  language=haskell,
  tabsize=2,
  emphstyle={\bf},
  commentstyle=\it,
  stringstyle=\mdseries\rmfamily,
  showspaces=false,
  keywordstyle=\bfseries\rmfamily,
  columns=flexible,
  basicstyle=\small\sffamily,
  showstringspaces=false,
  morecomment=[l]\%,
}
\title{Code generation}
\date{May 11, 2017}
\begin{document}
\maketitle

\section{Introduction}

\begin{frame}
\frametitle{Introduction}
In general we have chosen to keep it simple and do the ``obvious'' thing.
\begin{itemize}
\item Every value is always exactly 1 word.
 \begin{itemize}
 \item Characters, integers and booleans (and in a sense void) are reprsented as unboxed values.
 \item Pairs and lists are represented as pointers to heap objects.
 \end{itemize}
\item Code for expressions always put results on the stack.
\item Lists are linked lists as suggested by the syntax.
\item Stack management is done how the build in convenience instructions do it.
\end{itemize}
\end{frame}

\section{Semantics}

\begin{frame}
\frametitle{Semantics}

\begin{itemize}
\item Call by value.
\item But lists and pairs are always pointers, so they are ``references passed by value''.
        like objects in Java.
\end{itemize}
\end{frame}

\section{Calling convention}

\begin{frame}[fragile]
\frametitle{Stack frame layout}
\begin{columns}[c]
\column{.3\textwidth}
\begin{tabular}{l | c |}
  & arg 1 \\
\cline{2-2}
  & arg 2 \\
\cline{2-2}
  & ... \\
\cline{2-2}
  & arg $n$ \\
\cline{2-2}
  & ret. addr.  \\
\cline{2-2}
MP& old MP \\
\cline{2-2}
  & local $n$ \\
\cline{2-2}
  & ... \\
\cline{2-2}
  & local 2\\
\cline{2-2}
SP& local 1
\end{tabular}
\column{.7\textwidth}
The caller is responsible for putting from arg to ret. addr. on the stack.

\begin{block}{Prolouge}
\begin{lstlisting}
functionname: nop
link x
\end{lstlisting}
where x is the number of local variables.
\end{block}

\begin{block}{Epilouge}
\begin{lstlisting}
functionname_end: nop
unlink
ret
\end{lstlisting}
restores everything the prolouge did plus ret. addr.
\end{block}
\end{columns}
\end{frame}

\begin{frame}[fragile]
\frametitle{Calling convention}
\begin{columns}[c]
\column{.7\textwidth}
\begin{block}{Arguments}
\begin{itemize}
\item Caller pushes arguments on stack, then ret. addr.
\item Callee cleans up ret. add.
\item Caller cleans up arguments.
\end{itemize}
\end{block}
\begin{block}{Return values}
\begin{itemize}
\item At return statement, callee puts return value in RR. Then jumps to functionname_end.
\item After cleaning up arguments, caller pushes RR (unless statement).
\end{itemize}
\end{block}
\column{.3\textwidth}
Call site looks like:
\begin{lstlisting}
;push arg 1
...
;push arg n
bsr functionname
asj -n
ldr RR
\end{lstlisting}
\end{columns}
\end{frame}

\section{Lists and Tuples}

\begin{frame}{Lists and Pairs}
\begin{block}{Pairs}
The syntactic form \lstinline{(e1, e2)} allocates a new heap objects of
size 2 and stores the result of \lstinline{e1} and \lstinline{e2} in it.
\end{block}
\begin{block}{Lists}
The syntactic form \lstinline{[]} becomes to constant 0.

The syntactic form \lstinline{e1 : e2} allocates a new heap object of 
size 2 and stores the result of \lstinline{e1} and \lstinline{e2} in it.
\end{block}
\end{frame}

\section{Polymorphism}
\begin{frame}{Polymorphism}
Everything is 1 word. No problems(with code generation).
\end{frame}
\section{Overloading}

\begin{frame}{Overloading}
Not yet implemented. Current idea is:
\begin{itemize}
\item While doing type inference
 \begin{itemize}
 \item add concretized type to every function application in AST.
 \end{itemize}
\item While generating code for function applications
 \begin{itemize}
 \item Apply resulting substitution (from type inference) to annotated type.
 \item Use resulting type to choose a print implementation (error if none applicable).
 \end{itemize}
\end{itemize}
\end{frame}

\section{Conclusion}

\begin{frame}{Things missing}
\begin{itemize}
\item Overloading
\item Names (of labels).
\item Read currently reads input backwards.
\end{itemize}
\end{frame}

\end{document}
