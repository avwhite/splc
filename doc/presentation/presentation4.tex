\documentclass{beamer}

\usepackage{listings}
\usepackage{syntax}
\usepackage{amsmath}

\lstset{
  frame=none,
  xleftmargin=2pt,
  stepnumber=1,
  numbers=left,
  numbersep=5pt,
  numberstyle=\ttfamily\tiny\color[gray]{0.3},
  belowcaptionskip=\bigskipamount,
  captionpos=b,
  escapeinside={*'}{'*},
%  language=haskell,
  tabsize=2,
  emphstyle={\bf},
  commentstyle=\it,
  stringstyle=\mdseries\rmfamily,
  showspaces=false,
  keywordstyle=\bfseries\rmfamily,
  columns=flexible,
  basicstyle=\small\sffamily,
  showstringspaces=false,
  morecomment=[l]\%,
}
\title{Garbage Collection}
\date{June 15, 2017}
\begin{document}
\maketitle


\begin{frame}{Copying GC}
\begin{itemize}
\item Split heap into two equal size parts. Only use one at a time
\item At collection change which part is active.
\item Reallocate all live heap objects referenced from the stack.
\item Scan through moved objects, and also copy objects referenced by these
\item When moving object, store a forwarding pointer so it is only moved once.
\end{itemize}
\end{frame}

\begin{frame}{Requirements for code generation}
\begin{enumerate}
\item At runtime we must know if a given word contains a value or a
reference.
\item At runtime we have to know how many local variables 
including arguments are 
associated with a frame, and where they are.
\item A local can only contain a reference to a heap object if that
heap object is actually alive.
\end{enumerate}
\end{frame}

\section{Changes to code generation}

\begin{frame}{Heap object allocation}
The allocation function allocate heap object with two pieces of
date in the following way:
\begin{table}
\centering
\begin{tabular}{| c |}
\hline
unused \\
\hline
forwarding pointer \\
\hline
data \\
\hline
data \\
\hline
\end{tabular}
\end{table}
All pointers will point to odd address
\end{frame}
\begin{frame}{Tagging}
\begin{itemize}
\item References are now all numbered
\item We \emph{tag} values by multiplying them by two to make them all
even.
\begin{enumerate}
\item Untagging is divison by 2
\item tagging or untagging a reference is a no op.
\end{enumerate}
\item When doing operations on values we need them untagged.
\item When stored in memory (stack or heap) they must be tagged.
\end{itemize}
\end{frame}

\begin{frame}{Tagging - implementation}
\begin{itemize}
\item One approach: Have expressions deal with untagged values, and
tag right before storing.
\begin{itemize}
\item Problem: Type information is needed.
\item unnecesary tagging and untagging.
\end{itemize}
\item Code generation for expressions take a boolean indicating if 
they should produce a tagged or an untagged value.
\begin{itemize}
\item This way everything works smoothly with no type information.
\item We can even completely avoid trying to untag references (So
 we need no check, just divide by 2)
\end{itemize}
\end{itemize}
\end{frame}

\begin{frame}[fragile]
\frametitle{Updated Stack frame layout}
\begin{columns}[c]
\column{.3\textwidth}
\begin{tabular}{l | c |}
  & arg $n$ \\
\cline{2-2}
  & ... \\
\cline{2-2}
  & arg 2 \\
\cline{2-2}
  & arg 1 \\
\cline{2-2}
  & ret. addr.  \\
\cline{2-2}
MP& old MP \\
\cline{2-2}
  & $m+n$ \\
\cline{2-2}
  & local $m$ = 0 \\
\cline{2-2}
  & ... \\
\cline{2-2}
  & local 2 = 0\\
\cline{2-2}
  & local 1 = 0 \\
\cline{2-2}
  & copy of arg 1 \\
\cline{2-2}
  & copy of arg 2 \\
\cline{2-2}
  & ... \\
\cline{2-2}
SP& copy of arg $n$ 
\end{tabular}

\column{.7\textwidth}

\begin{block}{Prolouge}
\begin{enumerate}
\item Function label
\item Link $m + n + 1$
\item Store $m + n$
\item Copy args into locals
\item Zero all other locals.
\end{enumerate}

\end{block}

\begin{block}{Epilouge}
\begin{enumerate}
\item End label
\item Unlink
\item Ret
\end{enumerate}
\end{block}
\end{columns}
\end{frame}
\end{document}
