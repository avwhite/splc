\documentclass{beamer}

\usepackage{listings}
\usepackage{syntax}
\usepackage{amsmath}

\lstset{
  frame=none,
  xleftmargin=2pt,
  stepnumber=1,
  numbers=left,
  numbersep=5pt,
  numberstyle=\ttfamily\tiny\color[gray]{0.3},
  belowcaptionskip=\bigskipamount,
  captionpos=b,
  escapeinside={*'}{'*},
%  language=haskell,
  tabsize=2,
  emphstyle={\bf},
  commentstyle=\it,
  stringstyle=\mdseries\rmfamily,
  showspaces=false,
  keywordstyle=\bfseries\rmfamily,
  columns=flexible,
  basicstyle=\small\sffamily,
  showstringspaces=false,
  morecomment=[l]\%,
}
\begin{document}

\section{Introduction}

\begin{frame}
\frametitle{Overview}
\begin{enumerate}
\item Features
\item Abstractions used.
\item Type inference for statements (returns).
\item What is not working.
\end{enumerate}
\end{frame}

\section{Features}

\begin{frame}
\frametitle{Features}
\begin{itemize}
\item Type inference
\item Let polymorphism
\item Let bindings allow for recursion
\begin{itemize}
\item But no mutual recursion. Things can only refer to themselves and previously defined things.
\end{itemize}
\item Functions are required by type system to return a value.
\begin{itemize} 
\item Returns automatically inserted when ``missing''.
\end{itemize}
\end{itemize}
\end{frame}

\section{Substitution and Inference in Haskell}

\begin{frame}
\frametitle{Substitution and Inference in Haskell}
We have to do a lot of work with substitutions and inference for this part of the compiler, so we come up with abstractions for easing this.
\begin{itemize}
\item A typeclass for things we can perform substitution on
\item A monad for doing inference
\end{itemize}
\end{frame}

\begin{frame}[fragile]
\frametitle{Inference Monad}
During type inference we need to keep track of
\begin{itemize}
\item A context
\item The next fresh variable
\item Information about returns (ignore this for now)
\end{itemize}

For this we use the following monad transformer stack:

\begin{lstlisting}
data InfState = InfState {
    ctx :: [TypeContext],
    freshNum :: Int,
    rets :: [Bool]
    } deriving (Show)

data TypeError = NoUnify Type Type | NotInContext Identifier deriving (Show)

type Inference a = StateT InfState (Either TypeError) a
\end{lstlisting}
\end{frame}

\begin{frame}[fragile]
\frametitle{Inference Monad}

This allows us to write code like:
\begin{lstlisting}
typeInferExp (Op2E o e1 e2) ctx t = do
    s1 <- typeInferExp e1 ctx (opInType o)
    s2 <- typeInferExp e2 (subst s1 ctx) (opInType o)
    s3 <- lift $ mgu (subst (s1 <> s2) t) (opOutType o)
    pure (s1 <> s2 <> s3)
\end{lstlisting}
Note: This is from an older version where contexts where not part of the state, but passed explicitly.
\end{frame}


\begin{frame}[fragile]
\frametitle{Inference Monad}
We also have functions for dealing with the context (stack)

\begin{lstlisting}
ctxLookup :: String -> Inference TypeScheme
ctxAdd :: (String, TypeScheme) -> Inference ()
pushCtx :: Inference ()
popCtx :: Inference ()
getCtx :: Inference TypeContext
\end{lstlisting}
\end{frame}

\begin{frame}[fragile]
\frametitle{Substitutions}
Looking at the unification and type inference algorithms, we observe that we have to perform substitutions on many things. e.g:

\[
* = \mathcal{M} (\Gamma ^{*_1}, e_2, \sigma^{*_1}) \circ *_1
\]

Writing this in a way more like the following would be cool (maybe?):

\[
* = \mathcal{M}^{*_1}(\Gamma, e_2, \sigma)
\]

\begin{lstlisting}
class Substable a where
    subst :: Substitution -> a -> a
\end{lstlisting}
\end{frame}

\begin{frame}[fragile]
\frametitle{Substitutions}

Pretty much everything can be substituted. Uninteresting implementations omitted.

\begin{lstlisting}
instance Substable Type where ...
instance Substable TypeContext where ...
instance Substable TypeScheme where ...

instance Substable Substitution where
    subst s1 s2 = s1 <> s2
instance (Substable a, Substable b) => Substable (a -> b) where
    subst s f = \a -> subst s (f (subst s a))

instance Substable InfState where ...
instance (Substable a) => Substable (Either e a) where ...
instance (Substable a, Substable s, Functor m)
        => Substable (StateT s m a) where
    subst s = fmap (subst s) . withStateT (subst s)
\end{lstlisting}
\end{frame}

\begin{frame}[fragile]
\frametitle{Substitutions}
Now we can write our example like this:
\begin{lstlisting}
typeInferExp (Op2E o e1 e2) t = do
    s1 <- typeInferExp e1 (opInType o)
    s2 <- (subst s1 (typeInferExp e2)) (opInType o)
    lift $ (subst s2 mgu) t (opOutType o)
\end{lstlisting}
With the following definiton
\begin{lstlisting}
typeInferExp' e t s = subst s (typeInferExp e) t
\end{lstlisting}
We can even do this:
\begin{lstlisting}
typeInferExp (Op2E o e1 e2) t =
    typeInferExp  e1 (opInType o) >>=
    typeInferExp' e2 (opInType o) >>=
    lift . (mgu' t (opOutType o))
\end{lstlisting}
\end{frame}

\section{Type inference for statements}

\begin{frame}[fragile]
\frametitle{Statements}
Example:
\begin{lstlisting}
typeInferStmtList ((AssignS id fs e):ss) t = do
    schm@(TypeScheme bounds _) <- ctxLookup id
    vs <- replicateM (length bounds) freshVar
    s <- typeInferExp e (concrete schm vs)
    typeInferStmtList' ss t s
\end{lstlisting}
\end{frame}

\begin{frame}
\frametitle{Returns}
A challenge that is unique to statements, and is not easy to express in terms of type inference on a $\lambda$-calculus like language is return statements.
\begin{enumerate}
\item All returns must be of the same type.
\item Every function must return something.
\end{enumerate}
1 is easy. Just let the type of return statements be the type of the expressions the return, and ignore the type of other statements.

2 requires something extra. If we don't make sure of this:
\begin{itemize}
\item Functions with no annotations and no returns will have return type $\forall a. a$ due to the way we handle 1. This is a runtime error.
\item Will this be a problem for functions with return type restricted by type annotations?
\end{itemize}
\end{frame}

\begin{frame}[fragile]
\frametitle{Making sure functions return}
Solution: Keep a stack of booleans in the state.
\begin{itemize}
\item Initially the stack is \lstinline{[False]}.
\item When a return is encountered set the top element to \lstinline{True}.
\item When entering a branch of a control structure push False on the stack.
\item When leaving \emph{the last} branch of a control structure pop the number of branches elements from the stack. Then:
\begin{itemize}
  \item let $x$ be the conjunction of the popped elements.
  \item If one of the branches is guarenteed to be executed set the top element to be the disjunction of itself and $x$.
\end{itemize}
\end{itemize}
\end{frame}

\begin{frame}[fragile]
\frametitle{Returns}
The Inference
\begin{lstlisting}
implicitReturnNeeded :: Inference Bool
\end{lstlisting}
returns(infers?) true only if the stack is exactly \lstinline{[False]}. This means we are at the top level of a function and a return is not guarenteed.
\end{frame}

\begin{frame}[fragile]
\frametitle{Returns -- Example 1}
\begin{lstlisting}
typeInferStmtList (IfS e body1 (Just body2):ss) t = do
    s1 <- typeInferExp e TBool
    enterControl
    pushCtx
    s2 <- typeInferStmtList' body1 t s1
    popCtx
    enterControl
    pushCtx
    s3 <- typeInferStmtList' body2 t s2
    popCtx
    leaveControl 2 True
    typeInferStmtList' ss t s3
\end{lstlisting}
\end{frame}

\begin{frame}[fragile]
\frametitle{Returns -- Example 2}
\begin{lstlisting}
typeInferStmtList [] t = do
    b <- implicitReturnNeeded
    if b then
        lift $ mgu t TVoid
    else
        pure mempty
\end{lstlisting}
\end{frame}
\section{What is missing}

\begin{frame}
\frametitle{Things which are missing}

\begin{itemize}
\item Type annotations
\end{itemize}
\end{frame}

\end{document}
